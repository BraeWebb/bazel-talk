\documentclass[aspectratio=169]{beamer}

\usepackage{listings}
\definecolor{codegreen}{rgb}{0,0.6,0}
\definecolor{codegray}{rgb}{0.5,0.5,0.5}
\definecolor{codepurple}{rgb}{0.58,0,0.82}
\definecolor{backcolour}{rgb}{0.95,0.95,0.92}

\lstdefinestyle{language_style}{
    % backgroundcolor=\color{backcolour},   
    commentstyle=\color{codegreen},
    keywordstyle=\color{magenta},
    numberstyle=\tiny\color{codegray},
    stringstyle=\color{UQCSBlue},
    basicstyle=\ttfamily\footnotesize,
    breakatwhitespace=false,         
    breaklines=true,                 
    captionpos=b,                    
    keepspaces=true,                 
    numbers=left,                    
    numbersep=5pt,                  
    showspaces=false,                
    showstringspaces=false,
    showtabs=false,                  
    tabsize=2
}
\lstset{style=language_style}
\lstnewenvironment{bazel}{\lstset{language=python}}{}

\usepackage{slides/uqcs}

\title{Building with Bazel}
\author{Brae Webb}
\institute{University of Queensland}
\date{August 27, 2020}

% \theoremstyle{definition}
% \newtheorem{definition}{Definition}[section]

\begin{document}

% Slide types
% title-slide
% split-slide
% section-slide
% body-slide
% image-slide
% 4-pane-slide

\begin{title-slide}
\titlepage
\end{title-slide}

\begin{split-slide}{Outline}
    \vspace{4em}
    \tableofcontents
\end{split-slide}

\section{What is Bazel?}

\begin{body-slide}{Bazel 101}
\begin{itemize}[<+-| alert@+>]
    \color{UQCSBlue}
    \item Pronounced same as `basil' (the herb) in US English `BAY-zel'
    \item Build tool for many many languages
    \item Open-source of Google's internal tool Blaze
\end{itemize}

\end{body-slide}

\section{An Ad for Bazel}

\subsection{Bazel Features}
\begin{split-slide}{Overview}
\begin{enumerate}
    \item Isolated
    \item Cached
    \item Dependency graph
    \item Extendable
\end{enumerate}
\end{split-slide}

\begin{body-slide}{Isolated}
Builds are isolated to enforce accurate dependency specification
\end{body-slide}

\begin{body-slide}{Cached}
Builds specify what they output and that is cached

The cache can be local or a remote cache
\end{body-slide}

\begin{body-slide}{Dependency Graph}
All dependencies are as accurate as possible

Dependency graph only updates what is required

Allows isolated builds to build in parallel
\end{body-slide}

\begin{body-slide}{Extendable}
Add new rules for a language or framework in a subset of python
\end{body-slide}

\subsection{Bazel Languages}
{
\usebackgroundtemplate{%
\includegraphics[width=\paperwidth,height=\paperheight]{images/languages}
}
\begin{frame}
    
\end{frame}
}


\section{Actually Using Bazel}

\subsection{BUILD}

\begin{body-slide}{BUILD}
\begin{itemize}[<+-| alert@+>]
    \color{UQCSBlue}
    \item All the magic starts in a BUILD file
    \item Each 'package' in your repository will tend to have a BUILD
    \item BUILD files specify a list of builds using build rules
\end{itemize}
\end{body-slide}

\begin{body-slide}{Built-in Rules}
Bazel comes with built-in rules that you can use straight away

\only<2->{
\begin{enumerate}
    \item C/C++ rules
    \item Java rules
    \item Python rules
    \item Protobuf rules
\end{enumerate}
}
\end{body-slide}

\begin{body-slide}{Example}
\begin{bazel}
java_binary(
    name = "tree_serializer",
    srcs = [
        "SerializeTree.java",
    ],
    main_class = "SerializeTree"
)
\end{bazel}
\end{body-slide}


\end{document}