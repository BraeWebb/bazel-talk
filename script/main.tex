\documentclass{article}

\usepackage{xcolor}
\usepackage{listings}
\definecolor{codegreen}{rgb}{0,0.6,0}
\definecolor{codegray}{rgb}{0.5,0.5,0.5}
\definecolor{codepurple}{rgb}{0.58,0,0.82}
\definecolor{backcolour}{rgb}{0.95,0.95,0.92}

\lstdefinestyle{language_style}{
    % backgroundcolor=\color{backcolour},   
    commentstyle=\color{codegreen},
    keywordstyle=\color{magenta},
    numberstyle=\tiny\color{codegray},
    stringstyle=\color{codepurple},
    basicstyle=\ttfamily\footnotesize,
    breakatwhitespace=false,         
    breaklines=true,                 
    captionpos=b,                    
    keepspaces=true,                 
    numbers=left,                    
    numbersep=5pt,                  
    showspaces=false,                
    showstringspaces=false,
    showtabs=false,                  
    tabsize=2
}
\lstset{style=language_style}
\lstnewenvironment{bazel}{\lstset{language=python}}{}

\title{Bazel Presentation Notes}
\author{Brae Webb}
\date{August, 2020}

\begin{document}
    \maketitle

\begin{enumerate}
    \item Open java/tree\_serializer and explain SerializeTree
    \item Write a BUILD file for java\_binary
\begin{bazel}
java_binary(
    name = "tree_serializer",
    srcs = [
        "SerializeTree.java",
    ],
    main_class = "SerializeTree"
)
\end{bazel}
    \begin{enumerate}
        \item the rule is java\_binary
        \item the specific instructions here create a target using java\_binary
        \item every target needs a name
        \item other parameters vary and can be found in the docs
    \end{enumerate}
    \item Explain 3 modes of execution (slide)
    \item bazel build tree\_serializer in same directory
    \item bazel build //example/java/tree\_serializer:tree\_serializer from anywhere
    \item bazel build //example/java/tree\_serializer as shorthand when target
    name equals directory name
    \item note that it logs artifacts
    \item artifacts are stored in bazel-bin
    \item break SerializeTree into InvalidFormat, Node and StringProcessor
    \item rewrite BUILD to include java\_library
\begin{bazel}
java_library(
    name = "node_lib",
    srcs = [
        "Node.java"
    ],
)

java_binary(
    name = "tree_serializer",
    srcs = glob([
        "*.java",
    ]),
    deps = [":node_lib"],
    main_class = "SerializeTree"
)    
\end{bazel}
    \item explain the glob function allows wildcard
    \item explain that : means that it is a reference to another target
\end{enumerate}
\end{document}